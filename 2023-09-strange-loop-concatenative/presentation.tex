\documentclass[aspectratio=169]{beamer}
\mode<presentation>

\usepackage{adjustbox}
\usepackage{array}
\usepackage{calc}
\usepackage{fancyvrb}
\usepackage{relsize}
\usepackage{graphicx}
\usepackage[most]{tcolorbox}
\usepackage[overlay,absolute]{textpos}
\usepackage{xcolor}

\graphicspath{{../images}}

\usepackage{fontspec}
\setsansfont{Helvetica Neue Light}[
    BoldFont={Helvetica Neue Bold},
    BoldItalicFont={Helvetica Neue Bold Italic},
    ItalicFont={Helvetica Neue Light Italic}
]
\setmonofont{Iosevka Term}
\newfontface\helvreg{Helvetica Neue}

\definecolor{grayish}{gray}{0.8}

\title{Concatenative programming\\and stack-based languages}
\author{Douglas Creager}
\institute{Walland Heavy Research}
\date{Strange Loop\\\textsmaller[1]{September 21–22, 2023 – St. Louis}}

\setbeamercolor{title}{fg=black}
\setbeamerfont{title}{series=\bfseries,size=\larger[1]}
\setbeamerfont{subtitle}{series=\mdseries,size=\smaller[2]}
\setbeamerfont{author}{size=\smaller[1]}
\setbeamerfont{institute}{size=\smaller[2]}
\setbeamertemplate{navigation symbols}{}
\setbeamercolor{frametitle}{fg=black}
\setbeamerfont{frametitle}{series=\bfseries}

% Picture credits

\makeatletter
\def\picturecredits{}
\newcommand{\picturecredit}[4]{
    \protected@xappto\picturecredits{
        \textsmaller[3]{Slide \theframenumber} &
        \textsmaller[2]{#1, “#2”} \vspace*{-0.4em} \newline
        \textsmaller[3]{#3, \url{#4}} \\
    }
}
\makeatother

\newlength{\titlewidth}
\newcommand{\flattitle}[3]{
    \settowidth{\titlewidth}{\textbf{\LARGE #3}}
    \begin{textblock*}{\titlewidth}(#1,#2)
        \textbf{\LARGE #3}
    \end{textblock*}
}
\newcommand{\shadowedtitle}[3]{
    \settowidth{\titlewidth}{\textbf{\LARGE #3}}
    \addtolength{\titlewidth}{0.5mm}
    \begin{textblock*}{\titlewidth}(#1+0.4mm,#2+0.4mm)
        \textbf{\LARGE #3}
    \end{textblock*}
    \begin{textblock*}{\titlewidth}(#1,#2)
        \textbf{\textcolor{white}{\LARGE #3}}
    \end{textblock*}
}

%%%%%%%%%%%%%%%%%%%%%%%%%%%%%%%%%%%%%%%%%%%%%%%%%%%%%%%%%%%%%%%%%%%%%%%%%%%%%%%%
% Simple stack language

\newdimen\origiwspc
\origiwspc=\fontdimen2\font

\newcommand<>{\stackex}[2]{%
    \begin{onlyenv}#3
    \begin{tabular}[c]{
        @{\extracolsep{1em}}
        >{\rule[-0.6\baselineskip]{0pt}{1.8\baselineskip}\raggedleft\arraybackslash\fontdimen2\font=1ex}p{0.3\textwidth}<{\fontdimen2\font=\origiwspc}
        |
        >{\strut\raggedright\arraybackslash\leavevmode\color{blue}\ttfamily}p{0.6\textwidth}
    }
        \cline{1-1}
        #1 & #2 \\
        \cline{1-1}
    \end{tabular}
    \end{onlyenv}%
    \ignorespaces
}

\newcommand{\sym}[1]{\textcolor{blue}{\texttt{#1}}}

\newcommand{\stacksplit}[2][]{%
    \tcbox[
        standard jigsaw,
        nobeforeafter,
        size=fbox,
        boxrule=0pt,
        sharp corners=all,
        #1
    ]{\sym{\strut#2}}%
    \ignorespaces
}

%%%%%%%%%%%%%%%%%%%%%%%%%%%%%%%%%%%%%%%%%%%%%%%%%%%%%%%%%%%%%%%%%%%%%%%%%%%%%%%%

\begin{document}

\begin{frame}
    \titlepage
\end{frame}

%%%%%%%%%%%%%%%%%%%%%%%%%%%%%%%%%%%%%%%%%%%%%%%%%%%%%%%%%%%%%%%%%%%%%%%%%%%%%%%%
% Initial stack language examples

\begin{frame}
    \frametitle{Programs operate on a stack}
    \stackex<+>{       }{ 1 2 + }
    \stackex<+>{      1}{ 2 + }
    \stackex<+>{    1 2}{ + }
    \stackex<+>{      3}{ }
\end{frame}

\begin{frame}
    \frametitle{The stack doesn't have to start empty}
    \stackex<+>{    1337 42}{ 1 2 + }
    \stackex<+>{  1337 42 1}{ 2 + }
    \stackex<+>{1337 42 1 2}{ + }
    \stackex<+>{  1337 42 3}{ }
\end{frame}

\begin{frame}
    \frametitle{Stack underflow is not fatal}
    \stackex<+>{         }{ 2 + }
    \stackex<+>{        2}{ + }
    \stackex<+>{2 \sym{+}}{ }
\end{frame}

\begin{frame}
    \frametitle{Pythagoras}
    \stackex<+>{       }{ 1 2 + 3 * 2 2 + 7 3 - * + sqrt }
    \stackex<+>{      1}{ 2 + 3 * 2 2 + 7 3 - * + sqrt }
    \stackex<+>{    1 2}{ + 3 * 2 2 + 7 3 - * + sqrt }
    \stackex<+>{      3}{ 3 * 2 2 + 7 3 - * + sqrt }
    \stackex<+>{    3 3}{ * 2 2 + 7 3 - * + sqrt }
    \stackex<+>{      9}{ 2 2 + 7 3 - * + sqrt }
    \stackex<+>{    9 2}{ 2 + 7 3 - * + sqrt }
    \stackex<+>{  9 2 2}{ + 7 3 - * + sqrt }
    \stackex<+>{    9 4}{ 7 3 - * + sqrt }
    \stackex<+>{  9 4 7}{ 3 - * + sqrt }
    \stackex<+>{9 4 7 3}{ - * + sqrt }
    \stackex<+>{  9 4 4}{ * + sqrt }
    \stackex<+>{   9 16}{ + sqrt }
    \stackex<+>{     25}{ sqrt }
    \stackex<+>{      5}{ }
\end{frame}

%%%%%%%%%%%%%%%%%%%%%%%%%%%%%%%%%%%%%%%%%%%%%%%%%%%%%%%%%%%%%%%%%%%%%%%%%%%%%%%%
% Concatenation and composition

\begin{frame}
    \begin{onlyenv}<+>
    \begin{center}
        \stacksplit[opacityback=0]{1 2 + 3 * 2}
        \stacksplit[opacityback=0]{2 + 7 3 - *}
        \stacksplit[opacityback=0]{+ sqrt}
    \end{center}
    \end{onlyenv}

    \begin{onlyenv}<+>
    \begin{center}
        \stacksplit[colback=red!20]{1 2 + 3 * 2}
        \stacksplit[colback=gray!20]{2 + 7 3 - *}
        \stacksplit[colback=yellow!20]{+ sqrt}
    \end{center}
    \end{onlyenv}
\end{frame}

%%%%%%%%%%%%%%%%%%%%%%%%%%%%%%%%%%%%%%%%%%%%%%%%%%%%%%%%%%%%%%%%%%%%%%%%%%%%%%%%

\begin{frame}[t]
    \frametitle{Picture credits}
    \begin{tabular}{lp{0.75\textwidth}}
    \picturecredits
    \end{tabular}
\end{frame}


\end{document}
